\chapter{جمع‌بندی، نتیجه‌گیری و پیشنهادات برای کارهای آتی}
\section{جمع‌بندی و نتیجه‌گیری}
در این پروژه، سعی کردیم که یک دوقلوی دیجیتال از خیابان رشت بسازیم. در این پروژه، انواع معماری‌های ابزارها با دقت بررسی شدند تا بتوان به یک روش عملی برای پیاده‌سازی این دوقلوی دیجیتال رسید. مهم‌ترین اجزای این پروژه، کتابخانه‌های توسعه نرم‌افزار \lr{ROS2} و \lr{Autoware} بودند. بدون دانستن معماری دقیق این دو ابزار، این پروژه امکان پذیر نبود. فرایند گرفتن الهام و ایده برای استفاده از تشخیص‌دهنده مولفه ادراک \lr{Autoware}، نشان می‌دهد که برای حل کردن مسائل بزرگ، نیاز است تا آن را به زیرمسئله‌های کوچکتر شکاند. همچنین می‌توان نتیجه گرفت که برخی اوقات جواب داده شده برای یک مسئله بزرگ، ممکن است جوابی برای مسئله‌های بزرگ دیگر نیز باشد؛ به طور مثال، مسئله تشخیص اجسام در خودرو‌های خودران همانند مسئله تشخیص اجسام در پیاده‌سازی دوقلوی دیجیتال یک منطقه است.
در آخر نیز توجه داشته باشیم که شکل‌گیری این دوقلوی دیجیتال، به کمک شبیه‌ساز \lr{AWSIM} امکان‌پذیر شد و بدون افزونه آن که \lr{R2FU} نام داشت، راهی برای دریافت اطلاعات ردیاب در موتور بازی‌سازی \lr{Unity} وجود نداشت.
در فصل ارزیابی نیز به مشکلات این دوقلوی دیجیتال پرداخته که حتما نیاز به تحقیق و بررسی دارند. با وجود اینکه نمی‌توان به نتیجه این پژوهش، برچسب دوقلوی دیجیتال دقیق زد، اما می‌توان آن را گامی مثبت در جهت به واقعیت رسیدن این رویا دانست. با توجه به نتایج گرفته شده، می‌توان برای آینده این فناوری امیدوار بود.

\section{پیشنهادات برای کارهای آتی}
این پروژه، قابلیت تکمیل و توسعه از ابعاد مختلفی را داراست. برخی از آنها نیاز به تحقیقات بیشتر دارند و برخی از آنها، نیازمند کار عملی زیاد است. 

\subsection{طراحی نقشه ابر نقاط}
برای اینکه برخی از اجسام مربوط به نقشه، به عنوان اجسام حاضر در صحنه تشخیص داده نشوند، نیاز است که یک نقشه ابر نقاط با ادغام اسکن‌های لایدار در طول خیابان رشت، ساخت. با داشتن نقشه ابر نقاط، می‌توانیم از فیلتر مربوط به آن در ابزار \lr{Autoware} استفاده کنیم. 

\subsection{تهیه نقشه برداری}
با وجود داشتن نقشه برداری از طریق نرم‌افزار \lr{OpenStreetMap}، نمی‌توان به آن بسنده کرد زیرا ابعاد دقیق لاین‌های خیابان را به درستی در نیاورده است. همچنین اطلاعات مربوط به نقشه ایران در این ابزار، کمی قدیمی و بروزرسانی نشده است. از آنجایی که یکی از فیلتر‌های ابزار \lr{Autoware}، مبتنی بر این نقشه است که تا حد خوبی مثبت‌های کاذب را از تشخیص‌ها فیلتر می‌کند. پس حتما به یک نقشه برداری دقیق از منظقه دانشگاه امیرکبیر نیاز خواهیم داشت. 

\subsection{مدل‌سازی کامل منطقه دانشگاه امیرکبیر}
مدل سه‌بعدی استفاده شده در این پروژه، خیلی ساده است و بسیاری از ساختمان‌ها در آن مدل‌سازی نشده‌اند. این کار نیاز به متخصصین مدل‌سازی، مهندسین نقشه‌کشی و معمارها دارد. طراحی مدل سه‌بعدی یک منطقه کاری بسیار دشوار و هزینه‌بر است و در حد این پژوهش نبود.

\subsection{استفاده از مدل‌های جدید}
مدل استفاده شده در گره ادراک ابزار \lr{Autoware}، مدل هوش مصنوعی \lr{CenterPoint} بود که در سال ۲۰۲۱ ساخته شده است. هم اکنون سال ۲۰۲۳ است و مدل‌های هوش مصنوعی بسیار قوی‌تری در زمینه تشخیص اجسام سه‌بعدی وجود دارد. می‌توان مدل استفاده شده در \lr{Autoware} را با این مدل‌ها جایگزین کرد تا نتایجی بهتر گرفت.