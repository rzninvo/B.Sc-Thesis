%% -!TEX root = AUTthesis.tex
% در این فایل، عنوان پایان‌نامه، مشخصات خود، متن تقدیمی‌، ستایش، سپاس‌گزاری و چکیده پایان‌نامه را به فارسی، وارد کنید.
% توجه داشته باشید که جدول حاوی مشخصات پروژه/پایان‌نامه/رساله و همچنین، مشخصات داخل آن، به طور خودکار، درج می‌شود.
%%%%%%%%%%%%%%%%%%%%%%%%%%%%%%%%%%%%
% دانشکده، آموزشکده و یا پژوهشکده  خود را وارد کنید

\faculty{دانشکده مهندسی کامپیوتر}
% گرایش و گروه آموزشی خود را وارد کنید
\department{}
% عنوان پایان‌نامه را وارد کنید
\fatitle{دوقلوی دیجیتالی صحنه ترافیکی مبتنی بر دوربین و لایدار با استفاده از \lr{ROS} و \lr{AWSIM}
\\[.75 cm]
}
% نام استاد(ان) راهنما را وارد کنید
\firstsupervisor{دکتر مهدی جوانمردی}
%\secondsupervisor{استاد راهنمای دوم}
% نام استاد(دان) مشاور را وارد کنید. چنانچه استاد مشاور ندارید، دستور پایین را غیرفعال کنید.
% \firstadvisor{دکتر مهدی همایون‌پور}
%\secondadvisor{استاد مشاور دوم}
% نام نویسنده را وارد کنید
\name{رهام }
% نام خانوادگی نویسنده را وارد کنید
\surname{زنده‌دل نوبری}
%%%%%%%%%%%%%%%%%%%%%%%%%%%%%%
\thesisdate{شهریور ۱۴۰۲}

% چکیده پایان‌نامه را وارد کنید
\fa-abstract{
    در حوزه‌ی سیستم‌های مدرن حمل و نقل، توسعه و آزمون سیستم‌های حمل و نقل هوشمند\LTRfootnote{Intelligent Transportation Systems (ITS)}، الگوریتم‌های مدیریت ترافیک\LTRfootnote{Traffic Management} و خودرو‌های خودران\LTRfootnote{Autonomous Vehicle} جزو مهم‌ترین موضوع‌های در حال تحقیق هستند. اطمینان از کارایی و ایمنی این سیستم‌ها، به توانایی شبیه‌سازی دقیق سناریو‌های ترافیکی دنیای واقعی وابسته است. این نیازمندی، به ایجاد مفهومی به نام دوقلوی دیجیتال\LTRfootnote{Digital Twin} منجر شده است که به عنوان یک ابزار قدرتمند در شبیه‌سازی‌های کامپیوتری به طور گسترده پیاده‌سازی می‌شود. دوقلوی دیجیتال به عنوان یک نسخه مجازی پویا از محیط فیزیکی عمل می‌کند و به عنوان یک سکوی بی‌نظیر در سرعت بخشیدن به پیشرفت فناوری خودرو‌های خودران و مدیریت ترافیک، خدمت می‌کند. علاوه بر کاربرد در خودروهای خودران، دوقلوهای دیجیتال در دامنه‌های گسترده‌تری از زمینه‌ها، از جمله سیستم‌های حمل‌ و نقل هوشمند و سیستم‌های مدیریت ترافیک نیز اهمیت دارند. این اهمیت به ویژه در هنگام ارزیابی عملکرد الگوریتم‌ها و استراتژی‌ها در یک سناریوی ترافیکی شبیه‌سازی شده اما نزدیک به واقعیت نمایان می‌شود. شبیه‌سازی‌های ترافیک، نقش بسزایی را در آموزش منطق رانندگی خودکار ایفا می‌کنند و به دانشمندان اطمینان می‌دهند که این خودروها قادر به سازگاری با شرایط ترافیکی پویا و ویژه هر منطقه هستند. به طور سنتی، ایجاد شرایط ترافیکی در شبیه‌سازی‌ها از طریق روش‌های دستی یا سنتز با استفاده از مدل‌های ریاضی صورت می‌گیرد. با این حال، این رویکردها اغلب در بازتولید دقیق پیچیدگی‌های منحصر به فرد و ویژه مناطق واقعی ترافیک‌خیز، ناکام می‌مانند. علاوه بر این، این روش‌ها نیاز به تلاش‌های کاری فراوان دارند.‌‌ این پژوهش به رویکردی نوآورانه و عملی برای مواجهه با این چالش‌ها پرداخته است و یک روش خودکار را پیشنهاد می‌دهد که با بهره‌گیری از داده‌های حسگری لایدار\LTRfootnote{Lidar Sensor} و استفاده از الگوریتم‌های هوش مصنوعی تشخیص اشیاء سه‌بعدی\LTRfootnote{3D Object Detection}، سناریو‌های مختلف ترافیک واقعی جهان را با دقت شبیه‌سازی کند.
}
%توجه: ‌در اعداد اعشاری قسمت صحیح و اعشار جا به جا باید نوشته شوند تا در متن به طور صحیح نمایش داده شوند. به عنوان مثال در متن بالا عدد ۲۹.۰۷ را جا به جا وارد کرده تا در متن به طور صحیح نمایش داده شود. 

% کلمات کلیدی پایان‌نامه را وارد کنید
\keywords{هوش‌ مصنوعی، دوقلوی دیجیتال، شبیه‌سازی ترافیک، خودروی خوردان، سیستم‌ مدیریت ترافیک}


\AUTtitle
%%%%%%%%%%%%%%%%%%%%%%%%%%%%%%%%%%
\vspace*{7cm}
\thispagestyle{empty}
\begin{center}
\includegraphics[height=5cm,width=12cm]{besm}
\end{center}